\documentclass{tufte-handout}
\usepackage{amsmath}
\pagestyle{empty}
\usepackage[utf8]{inputenc}
\usepackage{mathpazo}
\usepackage{booktabs}
\usepackage{microtype}
\usepackage{xcolor}

\usepackage{tikz}
\usetikzlibrary{matrix}
\usetikzlibrary{chains}
\usetikzlibrary{decorations}

\title{Gorilla--Sea Cucumber Hash Report}
\author{Elias Jørgensen and Simon Høiberg}


\begin{document}

\maketitle

  \subsection{Results}

  The following table gives the similarity between each pair of species as a number between 0 and 1, higher values meaning ``more similar''.
  We have used the hash function $h(S) = |\#S \mod{d}|$ (where \#S is the Java hashcode of string S) with  $d=10000$ and $k$-grams of length $k=20$.
  As can be seen, the species closest to us is the Gorilla.

  \bigskip\noindent
  {\small\sf
  \begin{tabular}{rcccccccccccc}
  \toprule
    &Human
    &Gorilla
    &Monkey
    &Horse
    &Deer
    &Pig
    &Cow
    &Gull
    &Trout
    &R. Cod
    &Lamprey
    &Sea Cuc.
    \\\midrule
  Human & 1 & 0.846 & 0.347 & 0.068 & 0.008 & 0.139 & 0.1 & 0.015 & 0.015 & 0.023 & 0.015 & 0.007 \\
  Gorilla & 0.846 & 1 & 0.216 & 0.03 & 0.015 & 0.139 & 0.138 & 0.008 & 0.015 & 0.023 & 0.015 & 0  \\
  Monkey & 0.347 & 0.216 & 1 & 0.053 & 0.008 & 0.14 & 0.093 & 0.054 & 0.008 & 0.015 & 0 & 0.007 \\
  Horse & 0.068 & 0.03 & 0.053 & 1 & 0.015 & 0.031 & 0.046 & 0.023 & 0.031 & 0.015 & 0.007 & 0.029 \\
  Deer & 0.008 & 0.015 & 0.008 & 0.015 & 1 & 0.015 & 0 & 0.008 & 0.023 & 0.008 & 0.022 & 0.014 \\
  Pig & 0.139 & 0.139 & 0.14 & 0.031 & 0.015 & 1 & 0.085 & 0.023 & 0 & 0.008 & 0.008 & 0 \\
  Cow & 0.1 & 0.138 & 0.093 & 0.046 & 0 & 0.085 & 1 & 0.023 & 0 & 0 & 0.008 & 0.007 \\
  Gull & 0.015 & 0.008 & 0.054 & 0.023 & 0.008 & 0.023 & 0.023 & 1  & 0.008 & 0.023 & 0.008 & 0.007 \\
  Trout & 0.015 & 0.015 & 0.008 & 0.031 & 0.023 & 0 & 0 & 0.008 & 1 & 0.008 & 0.008 & 0.015 \\
  R. Cod & 0.023 & 0.023 & 0.015 & 0.015 & 0.008 & 0.008 & 0 & 0.023 & 0.008 & 1 & 0 & 0.043 \\
  Lamprey & 0.015 & 0.015 & 0 & 0.007 & 0.022 & 0.008 & 0.008 & 0.008 & 0.008 & 0 & 1 & 0 \\
  Sea Cuc. & 0.007 & 0 & 0.007 & 0.029 & 0.014 & 0 & 0.007 & 0.007  & 0.015 & 0.043 & 0 & 1 \\
  \bottomrule
  \end{tabular}
  }


  \subsection{Tests}

  Our static method {\tt double cos\_angle(int[] p, int[] q)} computes the cosine of the angle of two vectors of the same length $d$.
  We have tested it on the following examples:

  \bigskip\noindent
{ \small\sf
  \begin{tabular}{cccc}
  \toprule
  $p$     & $q$     & $d$ & value returned \\\midrule
  $(0,1)$ & $(0,1)$ & $2$ & $1$ \\
  $(0,1)$ & $(0,2)$ & $2$ & $1$\\
  $(0,1)$ & $(1,0)$ & $2$ & $0$\\
  $(0,2)$ & $(0,2)$ & $2$ & $1$\\
  $(4,-3)$ & $(2,2)$ & $2$ & $0,1414$\\
  $(0,1)$ & $(0,-1)$ & $2$ & $0$\\
  \bottomrule
  \end{tabular}
}

\bigskip
  Similarly, our static method {\tt length(int[] p)} computes $[\ldots]$.
  \subsection{Performance}
  Our Java implementation took 0,04 seconds to compute the above on a late 2015 Macbook Pro.

\end{document}
