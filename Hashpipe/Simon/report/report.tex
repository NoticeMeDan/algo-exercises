\documentclass[a4paper,11pt]{article}

% Import packages
\usepackage[a4paper]{geometry}
\usepackage[utf8]{inputenc}
\usepackage{amsmath}
\usepackage{amssymb}
\usepackage{graphicx}
\usepackage{scrextend}
\usepackage{listings}
\usepackage{color}
\usepackage{adjustbox}

\definecolor{dkgreen}{rgb}{0,0.6,0}
\definecolor{gray}{rgb}{0.5,0.5,0.5}
\definecolor{mauve}{rgb}{0.58,0,0.82}

\lstset{frame=tb,
  language=Java,
  aboveskip=3mm,
  belowskip=3mm,
  showstringspaces=false,
  columns=flexible,
  basicstyle={\small\ttfamily},
  numbers=none,
  numberstyle=\tiny\color{gray},
  keywordstyle=\color{blue},
  commentstyle=\color{dkgreen},
  stringstyle=\color{mauve},
  breaklines=true,
  breakatwhitespace=true,
  tabsize=3
}

\graphicspath{ {images/} }
\DeclareGraphicsExtensions{.pdf,.jpeg,.png,.jpg}

% Change enumerate environments you use letters
\renewcommand{\theenumi}{\alph{enumi}}

% Set title, author name and date
\title{Hashpipe Report}
\author{Simon Høiberg}
\date{\today}

\begin{document}

\maketitle

\section*{Abstract}
\begin{addmargin}[1em]{2em}
HashPipe.java is a symbol table implemented with a Hashpipe data structure.\\
HashPipe supports search and insertion in time proportional to $O(Log N)$ on\\ average.\\
\hspace{5mm}\\
HashPipe contains the following API:\\
\begin{verbatim}
public HashPipe()
public int size()
public void put(String key, Integer val)
public Integer get(String key)
public String floor(String key)
 \end{verbatim}
\end{addmargin}




\section*{Class design}
\begin{addmargin}[1em]{2em}
HashPipe uses a pointer-based data structure where each key-value pair is represented by a pipe.\\
\hspace{5mm}\\
Every pipe has a given height thus contains a pointer to the neighbour pipe(s).\\
On instantiation, a root-pipe is created which solely represents pointers and every key-value pair that is put will be represented as new pipes.\\
\hspace{5mm}\\
The height of the root-pipe will be fixed at 32, and every new pipe will have it's height calculated deterministically.\\
\newpage
Every pipe is based in the following structure
\begin{lstlisting}
class Pipe {
		String key;
		int value;
		int height;
		Pipe[] nextPipes;

		// Root pipe will be created with fixed height of 32
		public Pipe()

		// New pipe will be created given it's key-value pair
		public Pipe(String key, int value)

		public String getKey()

		public int getValue()

		public int getHeight()

		public void setNextPipe(Pipe pipe, int index)

		public Pipe getNextPipe(int index)
	}
\end{lstlisting}
The field \textbf{Pipe[] nextPipes} will represent the height of the pipe with \textit{height} amount of pointers to the neighbour pipe(s).\\ 
\hspace{5mm}\\
The \textbf{put()} method works by following the root-pipe's pointers searching from top down and recursivly rearranging the pointers such that the key-value pipes will be ordered by key ascending fra left to right.\\
\hspace{5mm}\\
The \textbf{get()} method works by recursivly following the root-pipe's pointers searching from top down until a key-match is found or the bottom of the pipe is reached.\\
The value of the pipe will be returned if key-match is found, null is returned otherwise.
\end{addmargin}
\newpage

\section*{Performence tests}
Following performence tests is done using CodeJudge\\
\hspace{5mm}\\

\textbf{TestsSmall}\\
\hspace{5mm}\\
\begin{adjustbox}{width=1\textwidth}
\tiny
\begin{tabular}{cccc}
\hline
\textit{Test} & \textit{Status} & \textit{CPU TIME} & \textit{WALL TIME} \\
\hline
Test01 & Succeeded & 0.32 s & 0.42 s \\
Test02 & Succeeded & 0.17 s & 0.21 s \\
Test03 & Succeeded & 0.13 s & 0.18 s \\
\hline
\end{tabular}
\end{adjustbox}

\hspace{5mm}\\
\hspace{5mm}\\

\textbf{TestsBest}\\
\hspace{5mm}\\
\begin{adjustbox}{width=1\textwidth}
\tiny
\begin{tabular}{cccc}
\hline
\textit{Test} & \textit{Status} & \textit{CPU TIME} & \textit{WALL TIME} \\
\hline
Test01 & Succeeded & 0.88 s & 0.93 s \\
Test02 & Runtime error & 2.78s s & 2.97 s \\
\hline
\end{tabular}
\end{adjustbox}

\hspace{5mm}\\
\hspace{5mm}\\

\textbf{TestsBest}\\
\hspace{5mm}\\
\begin{adjustbox}{width=1\textwidth}
\tiny
\begin{tabular}{cccc}
\hline
\textit{Test} & \textit{Status} & \textit{CPU TIME} & \textit{WALL TIME} \\
\hline
Test01 & Succeeded & 0.54 s & 0.59 s \\
Test02 & Succeeded & 1.12s s & 1.17 s \\
\hline
\end{tabular}
\end{adjustbox}

\end{document}
